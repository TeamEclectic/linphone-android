\documentclass[11pt]{article}
\usepackage{graphicx}
\usepackage[bookmarks=true]{hyperref}
\usepackage{bookmark}
\usepackage{hyperref}
\usepackage{float}
\usepackage{wrapfig}

\usepackage{array}
\newcolumntype{L}[1]{>{\raggedright\let\newline\\\arraybackslash\hspace{0pt}}m{#1}}
\newcolumntype{C}[1]{>{\centering\let\newline\\\arraybackslash\hspace{0pt}}m{#1}}
\newcolumntype{R}[1]{>{\raggedleft\let\newline\\\arraybackslash\hspace{0pt}}m{#1}}

\begin{document}

\input{./title_page.tex}

\setcounter{tocdepth}{3}
\tableofcontents

\newpage
\section{Revision History}
\begin{table}[h]
\begin{tabular}{llll}
\textbf{Date}          & \textbf{Description}  & \textbf{Author}       & \textbf{Comments}   \\ \hline
\multicolumn{1}{|R{2.5cm}|}{26/05/2015} & \multicolumn{1}{L{5.5cm}|}{Document Creation} & \multicolumn{1}{l|}{Team Eclectic} & \multicolumn{1}{L{4cm}|}{Version 1} \\ \hline
\multicolumn{1}{|l|}{} & \multicolumn{1}{l|}{} & \multicolumn{1}{l|}{} & \multicolumn{1}{l|}{} \\ \hline
\multicolumn{1}{|l|}{} & \multicolumn{1}{l|}{} & \multicolumn{1}{l|}{} & \multicolumn{1}{l|}{} \\ \hline
\multicolumn{1}{|l|}{} & \multicolumn{1}{l|}{} & \multicolumn{1}{l|}{} & \multicolumn{1}{l|}{} \\ \hline
\multicolumn{1}{|l|}{} & \multicolumn{1}{l|}{} & \multicolumn{1}{l|}{} & \multicolumn{1}{l|}{} \\ \hline
\multicolumn{1}{|l|}{} & \multicolumn{1}{l|}{} & \multicolumn{1}{l|}{} & \multicolumn{1}{l|}{} \\ \hline
\multicolumn{1}{|l|}{} & \multicolumn{1}{l|}{} & \multicolumn{1}{l|}{} & \multicolumn{1}{l|}{} \\ \hline
\multicolumn{1}{|l|}{} & \multicolumn{1}{l|}{} & \multicolumn{1}{l|}{} & \multicolumn{1}{l|}{} \\ \hline
\multicolumn{1}{|l|}{} & \multicolumn{1}{l|}{} & \multicolumn{1}{l|}{} & \multicolumn{1}{l|}{} \\ \hline
\multicolumn{1}{|l|}{} & \multicolumn{1}{l|}{} & \multicolumn{1}{l|}{} & \multicolumn{1}{l|}{} \\ \hline
\end{tabular}
\end{table}

\section{Document Approval}
\begin{table}[h]
\begin{tabular}{llll}
\textbf{Signature}     & \textbf{Printed Name} & \textbf{Title}        & \textbf{Comments}     \\ \hline
\multicolumn{1}{|l|}{} & \multicolumn{1}{L{4.5cm}|}{} & \multicolumn{1}{L{4cm}|}{} & \multicolumn{1}{L{4cm}|}{} \\ \hline
\multicolumn{1}{|l|}{} & \multicolumn{1}{l|}{} & \multicolumn{1}{l|}{} & \multicolumn{1}{l|}{} \\ \hline
\multicolumn{1}{|l|}{} & \multicolumn{1}{l|}{} & \multicolumn{1}{l|}{} & \multicolumn{1}{l|}{} \\ \hline
\multicolumn{1}{|l|}{} & \multicolumn{1}{l|}{} & \multicolumn{1}{l|}{} & \multicolumn{1}{l|}{} \\ \hline
\multicolumn{1}{|l|}{} & \multicolumn{1}{l|}{} & \multicolumn{1}{l|}{} & \multicolumn{1}{l|}{} \\ \hline
\multicolumn{1}{|l|}{} & \multicolumn{1}{l|}{} & \multicolumn{1}{l|}{} & \multicolumn{1}{l|}{} \\ \hline
\multicolumn{1}{|l|}{} & \multicolumn{1}{l|}{} & \multicolumn{1}{l|}{} & \multicolumn{1}{l|}{} \\ \hline
\multicolumn{1}{|l|}{} & \multicolumn{1}{l|}{} & \multicolumn{1}{l|}{} & \multicolumn{1}{l|}{} \\ \hline
\multicolumn{1}{|l|}{} & \multicolumn{1}{l|}{} & \multicolumn{1}{l|}{} & \multicolumn{1}{l|}{} \\ \hline
\multicolumn{1}{|l|}{} & \multicolumn{1}{l|}{} & \multicolumn{1}{l|}{} & \multicolumn{1}{l|}{} \\ \hline
\end{tabular}
\end{table}

\newpage
\section{Introduction}

\subsection{Purpose}
\subsection{Scope}
\includegraphics[width=380px]{./images/scope.jpg} \newline
The scope of this documentation is to cover all use cases and functional requirements that are neccessary to provide group chat functionality to the Linphone service. \newline
The scope of ManageGroups is restricted to creating, deleting and removing groups. These are further split up into lower-level use cases that are needed in order to achieve the desired functionality.\newline
\subsection{Definitions, Acronyms \& Abbreviations}
\subsubsection{Internal Documentation}
\paragraph{FR:} Functional Requirement
\paragraph{UC:} Use Case
%\subsection{References}
\subsection{Overview}

\section{Specific Requirements}
%\subsection{External Interface Requirements}
%\subsubsection{User Interfaces}
%\subsubsection{Hardware Interfaces}
%\subsubsection{Software Interfaces}
%\subsubsection{Communication Interfaces}

\subsection{Functional Requirements}
\subsubsection{Group Chat Creation (critical)}
The ManageGroups module provides services to create, update and remove or leave groups.\newline
\textbf{Description:} The user activates the createGroup functionality by selecting an option to create a new group chat. The system will provide the user with a screen where details about the group such as group photo, group name and group members need to be specified. If all pre-conditions are met then the group is created. It allows for all added members to exchange messages and communicate in a common space.\newline
\newline
\begin{itemize}
\item ManageGroups.CreateGroup – Priority: Critical.
\item	This use case allows users to create a group where more than 1 contact can communicate simultaneously together.
\item	This creates a space where messages can be exchanged between users that are members of the groups.
\end{itemize}
\textbf{Services Contract:} \newline
	\newline
\begin{itemize}
\item The CreateGroupRequest identifies the user that is creating the group and automatically assigns this user to be the administrator.
\item It allows the user to select contacts that the user wishes to add to the group.
\item The group members are then notified by receiving an invitation to join and are linked to the group.
\item The service has three pre-conditions and three post-conditions
\end{itemize}
\includegraphics[width=380px]{./images/serviceContract-create.jpg} \newline
\textbf{Pre-conditions} \newline
For each of the pre-conditions below an exception is introduced which is raised by the service to notify the caller that the service is not being provided as the pre-condition associated with that exception has not been met.\newline
The group will not be created if one of the following scenarios occurs:\newline
\newline
\begin{itemize}
\item	User must have Linphone Service\newline
\item	No other group with the same name may exist \newline
\item	The user may not create the group unless there is at least one member/contact added.\newline
\end{itemize}
\textbf{Post-Conditions}\newline
The post-conditions specify the conditions which must hold true when the service has been provided.\newline
\newline
\begin{itemize}
\item	User is automatically assigned administrator rights
\item	An invite is sent to users to ask them if they want to be part of the group
\item	The group is created with the respective profiles and members
\end{itemize}
\textbf{Functional Requirements}\newline
 The functional requirements specify what the system should do and how the system should behave. Each use case has functional requirements and these specify the functions required by the use cases in order to use the service specified in the services contract. Each functional requirement is either checking for a post-condition or a precondition or both.\newline
 \newline
\includegraphics[width=380px]{./images/FR-create.jpg}
 \newline
\textbf{Process Specification} \newline
 \newline
 \includegraphics[width=380px]{./images/process-create.jpg}
  \newline
 First the user creating the group fills in the relative details such as the group name and adds members to the group. Each of these members needs to have the Linphone service to be added to the group. If the contacts do not have Linphone, an error is thrown. Next the user tries to create the group. If less than 1 member is added then an error is thrown and the group is not created. If more than one member was added then the system continues to check whether there are any other groups with the same name. If so, then an exception is thrown, otherwise the user is assigned as group administrator, an invite is sent to the members and the group is created.\newline
 \newline
 \textbf{Included Use cases} \newline
 Use cases that are included are necessary to provide the lower level functionality needed in order to achieve the higher level functionality. These all deal with the resources and retrieving from storage. They  are:\newline
 \newline
 \textbf{addGroupPhoto  – Priority: Important} \newline
 \begin{itemize}
 \item	This use case allows a member of the group and the administrator, on group creation, to add a group photo. On group creation, if the creator does not specify a group photo, a default one is assigned.\newline
 \end{itemize}
 \textbf{getContactList – Priority: Important} \newline
 \begin{itemize}
 \item	This use case gets the user trying to create the group’s contact list and checks whether each contact added to the group has Linphone or not. If not an exception is thrown otherwise the system proceeds further.\newline
 \end{itemize}
 \textbf{getGroups  – Priority: Critical} \newline
 \begin{itemize}
\item	This use case gets all the group names that the creator has in their chats list and checks that another group with the same name as the new group exists. If another group exists then an exception is thrown otherwise the system proceeds further. \newline
\end{itemize}
 \textbf{sendInvite – Priority: Critical} \newline
 This use case is triggered when a group is created and when new members are added to the group.  \newline 
\textbf{Services Contract:}
 \begin{itemize}
 \item	This use case contains the invite itself as well as information regarding where to send it etc.\newline
 \end{itemize}
\textbf{Pre-conditions} \newline
 For each of the pre-conditions below an exception is introduced which is raised by the service to notify the caller that the service is not being provided as the pre-condition associated with that exception has not been met.\newline
 The invite will not be sent if one of the following scenarios occurs:
 \begin{itemize}
 \item	The receiver of the invite is not a contact added by the administrator of the group.
 \end{itemize}
\textbf{Post-conditions} \newline
 The post-conditions specify the conditions which must hold true when the service has been provided.
 \begin{itemize}
\item	The invite must be sent to the correct user.
\end{itemize}
 \includegraphics[width=380px]{./images/serviceContract-sendInvite.jpg} \newline
\textbf{Functional Requirements}\newline
 The functional requirements specify what the system should do and how the system should behave. Each use case has functional requirements and these specify the functions required by the use cases in order to use the service specified in the services contract. Each functional requirement is either checking for a post-condition or a precondition or both.\newline
 \newline
\includegraphics[width=200px]{./images/FR-sendInvite.jpg}
 \newline
\textbf{Process Specification} \newline
 \newline
 \includegraphics[width=200px]{./images/process-sendInvite.jpg}
  \newline
 
\subsubsection{Delete Group Chat} \label{FR-delete-group}
\paragraph{Priority:}Critical
\paragraph{Summary:}
\paragraph{Rationale:}
\paragraph{Requirements:}
\paragraph{References:} UC \ref{FR-delete-group}
\subsubsection{Update Group Chat Profile} \label{FR-update-group}
\paragraph{Priority:}Important
\paragraph{Summary:}
\paragraph{Rationale:}
\paragraph{Requirements:}
\paragraph{References:} UC \ref{FR-update-group}
\subsubsection{Add Member} \label{FR-add-member}
\paragraph{Priority:}Critical
\paragraph{Summary:}
\paragraph{Rationale:}
\paragraph{Requirements:}
\paragraph{References:} UC \ref{FR-add-member}
\subsubsection{Remove Member} \label{FR-remove-member}
\paragraph{Priority:}Critical
\paragraph{Summary:}
\paragraph{Rationale:}
\paragraph{Requirements:}
\paragraph{References:} UC \ref{FR-remove-member}
\subsubsection{Send Message} \label{FR-send-message}
\paragraph{Priority:}Critical
\paragraph{Summary:}
\paragraph{Rationale:}
\paragraph{Requirements:}
\paragraph{References:} UC \ref{FR-send-message}
\subsubsection{Send Voice Recording} \label{FR-send-voice}
\paragraph{Priority:}Important
\paragraph{Summary:}
\paragraph{Rationale:}
\paragraph{Requirements:}
\paragraph{References:} UC \ref{FR-send-voice}

\subsection{Use Cases}
\subsubsection{Delete Group Chat} \label{UC-delete-group}
\paragraph{Summary:}
\paragraph{Rationale:}
\paragraph{Users:}
\paragraph{Preconditions:}
\paragraph{{Postconditions:}}
%\includegraphics[]{name} -- for use case diagram
\subsubsection{Update Group Chat Profile} \label{UC-update-group}
\paragraph{Summary:}
\paragraph{Rationale:}
\paragraph{Users:}
\paragraph{Preconditions:}
\paragraph{{Postconditions:}}
%\includegraphics[]{name} -- for use case diagram
\subsubsection{Add Member} \label{UC-add-member}
\paragraph{Summary:}
\paragraph{Rationale:}
\paragraph{Users:}
\paragraph{Preconditions:}
\paragraph{{Postconditions:}}
%\includegraphics[]{name} -- for use case diagram
\subsubsection{Remove Member} \label{UC-remove-member}
\paragraph{Summary:}
\paragraph{Rationale:}
\paragraph{Users:}
\paragraph{Preconditions:}
\paragraph{{Postconditions:}}
%\includegraphics[]{name} -- for use case diagram
\subsubsection{Send Message} \label{UC-send-message}
\paragraph{Summary:}
\paragraph{Rationale:}
\paragraph{Users:}
\paragraph{Preconditions:}
\paragraph{{Postconditions:}}
%\includegraphics[]{name} -- for use case diagram
\subsubsection{Send Voice Recording} \label{UC-send-voice}
\paragraph{Summary:}
\paragraph{Rationale:}
\paragraph{Users:}
\paragraph{Preconditions:}
\paragraph{{Postconditions:}}
%\includegraphics[]{name} -- for use case diagram

\subsection{Non-Functional Requirements}
%example
\subsubsection{Performance constraints for Add User} \label{NF-performance-add-user}
\paragraph{Summary:}
\paragraph{Rationale:}
\paragraph{Requirements:}
\paragraph{References:} FR \ref{FR-add-member}, UC \ref{UC-add-member}


%\section{Requirements Traceability Matrix}

\newpage
\section{Appendix A} \label{appendix-a}
\listoffigures

\end{document}